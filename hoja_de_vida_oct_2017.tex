\documentclass[line,margin]{res}

\usepackage[utf8]{inputenc}
\usepackage[spanish]{babel}

\begin{document}

\name{Iván Mauricio Burbano Aldana}
\address{ivanmbur@gmail.com\\ Cr. 19 63 27, Bogotá, Colombia\\ (+57) 316 782 1110}

\begin{resume}

\section{Sobre mí}

Nacido el 26 de diciembre de 1996 en Bucaramanga, Colombia, desde pequeño me distinguí por mis fuertes pasiones. Mis primeros descrubrimientos a través de la música, que después entendería fueron en realidad sobre matemáticas y simetría, asentaron mi amor por entender lo que nos rodea. Ahora, mediante la física matemática, he encontrado un camino con el cual puedo explorar el universo haciendo uso de mi creatividad. Estoy en un punto de mi vida en el cual estoy decidido a tomar provecho de todas las oportunidades que se me presenten en este camino para aprender nuevas cosas, conocer gente nueva y convertirme en una mejor persona. Estoy seguro que el éxito en esta empresa está asegurado por la diciplina y compromiso que tengo hacia mi trabajo.

\section{Habilidades Especiales}

{\sl Física teórica}: Estados KMS, agujeros negros regulares y métodos algebraicos, topológicos y geométricos en física. 

{\sl Matemáticas}: Álgebra, topología, análisis, teoría de la medida y topología algebraica.

{\sl Programación}: Unix/Linux, Java, Python, C, IRIS y DS9.

{\sl Física experimental}: Sensores infrarojos y grafeno exfoliado electroquímicamente.  

{\sl Lenguajes}: Español e inglés ({\sl TOEFL} 113/120). 3 semestres de estudios de alemán.

\section{Educación}
{\sl Físico} \\
Universidad de los Andes \\
Opción en matemáticas \\
Esperado en Marzo 2018 \\
Promedio: 4.73/5

{\sl Bachiller académico} \\
Colegio San Carlos \\
Junio 2014

{\sl Primaria} \\
Colegio la Quinta del Puente

\section{Reconocimientos}

\begin{itemize}

\item {\sl Distinción de Excelencia Semestral} por el mejor promedio semestral del departamento de Física de la Universidad de los Andes en el primer semestre del 2017.

\item {\sl Distinción Ramón de Zubiría} por el mejor promedio global del departamento de Física de la Universidad de los Andes en el primer semestre del 2016.

\item {\sl Distinción de Excelencia Semestral} por el mejor promedio semestral del departamento de Física de la Universidad de los Andes en el segundo semestre del 2015.

\item {\sl Mención de Honor} en la cuagragésima quinta Olimpiada Internacional de Física (Kazajistán 2014)

\item {\sl Tercer puesto} en la vigésima novena Olimpiada Colombiana de Física (2013)

\end{itemize}

\section{Experiencia}

{\sl SURF California Institute of Technology}: Trabajé bajo la dirección del Dr. Roger Smith y el Dr. Andrés Plazas durante el verano del 2017. Apoyé la investigación, caracterización y corrección del movimiento de centroides de imágenes en el Precision Projector Laboratory del Jet Propulsion Laboratory (NASA). Los detectores investigados serán utilizados para el estudio de lentes gravitacionales débiles y materia oscura en proyectos como WFIRST y Euclid.

{\sl Monitor Universidad de los Andes}: 
\begin{itemize}

\item Álgebra Lineal 2 con el profesor César Galindo en el segundo semestre del 2016.

\item Clínica de Problemas en el segundo semestre del 2016.

\item Álgebra Lineal Honores con el profesor Sergio Adarve en el primer semestre del 2016.

\end{itemize}

{\sl Laboratorio de nanomateriales}: Durante el segundo semestre del 2016 investigué propiedades actuadoras del grafeno bajo la dirección de la profesora Yenny Hernández como requisito para el curso de Laboratorio Intermedio.
  
{\sl Monitorias particulares}: He apoyado a estudiantes en Física 1, Física 2, Cálculo Integral, Cálculo Vectorial, Álgebra Lineal 2 y  Métodos Matemáticos. 

\section{Publicaciones}

{\sl KMS States and Tomita-Takesaki Theory}: Monografía de grado bajo la dirección del profesor Andrés Reyes (en curso).

{\sl Investigation, Characterization and Correction of Image Centroid Motion at JPL's Precision Projector Laboratory} (en curso)

\section{Seminarios}

{\sl Seminario de Teoría Cuántica de Campos/Física Matemática}: Retículos ortocomplementados cuánticos de proposiciones.

{\sl Lanzamiento Revista La Cicuta Séptima Edición}: Sobre la divulgación y censura de la ciencia. Fui el ponente invitado como joven investigador. 
  
\section{Actividades extracurriculares}

\begin{itemize}

\item Clases preparatorias en la Iglesia de Nuestra Señora de las Aguas para el exámen Saber 11 dirigida a personas de bajos recursos en el barrio La Candelaria.

\item Clases de guitarra dirigidas a personas de bajos recursos estudiantes de colegios de la Alianza Educativa. Esto fue parte de mi servicio social en bachillerato.

\item Escogido presidente del consejo estudiantil del colegio San Carlos durante mi último año de bachillerato.

\item Junto al Centro de Información de las Naciones Unidas diseñamos el primer modelo de Naciones Unidas a nivel distrital SIMONU 2013. Además, fuí Secretario General del modelo del Colegio San Carlos SACMUN X y me desempeñé como presidente y delegado en varios modelos de otros colegios.

\end{itemize}  
  
\section{Referencias}  

\begin{itemize}

\item Prof. Andrés Fernando Reyes Lega: anreyes@uniandes.edu.co

\item Dr. Roger Smith: rsmith@astro.caltech.edu 

\item Dr. Andrés Plazas: andres.a.plazas.malagon@jpl.nasa.gov

\item Prof. Carlos Andrés Flórez Bustos: ca.florez@uniandes.edu.co

\item Prof. Sergio Adarve: sadarve@uniandes.edu.co

\item Prof. César Neyit Galindo Martínez: cn.galindo1116@uniandes.edu.co

\end{itemize}
  
\end{resume}
\end{document}
