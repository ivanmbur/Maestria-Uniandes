\documentclass{letter}

\usepackage[utf8]{inputenc}
\usepackage[spanish]{babel}
\usepackage{setspace}
\doublespacing

\signature{Iván Mauricio Burbano Aldana \\ CC. 1019126827}
\address{Cr 19 63 27 \\ Bogotá, Colombia}

\begin{document}

\begin{letter}{Comité de Maestría y Doctorado \\ Departamento de Física \\ Universidad de los Andes}

\opening{Estimado comité:}

Durante mis estudios de bachillerato me enamoré por la física. Esto llevo a mi participación en la Olimpiadas Colombianas de Física donde obtuve el tercer puesto y, posteriormente, a que representara a Colombia en la Olimpiadas Internacionales de Física donde obtuve una mención de honor. Esta experiencia me permitió acceder a un entrenamiento preparatorio intensivo dictado por la Universidad Antonio Nariño que me dió una ventaja inicial en mis estudios de pregrado. Continué mis estudios ingresando al programa de Pregrado en Física de la Universidad de los Andes. Durante este, descubrí un interés particular por las matemáticas que me llevó a realizar una opción en esta area. Ya motivado por intereses teóricos, esta opción me permitió empezar a investigar en el area de Física Matemática. Mis intereses se han enfocado en la implementación de métodos algebraicos, topológicos y geométricos en física. En particular, estoy realizando mi proyecto de grado bajo la dirección de profesor Andrés Fernando Reyes Lega en el tema de estados KMS y teoría de Tomita-Takesaki. Aún más, estoy realizando un proyecto de investigación en el tema de agujeros negros regulares que empezó como parte del curso de Relatividad General dictado por el profesor Pedro Bargueño. Paralelo a esto, tuve la oportunidad de ingresar al programa SURF del California Institute of Technology. Durante el verano del 2017 estuve investigando bajo la dirección del doctor Roger Smith y otros científicos del Jet Propulsion Laboratory (NASA), el movimiento de centroides de imágenes en sensores infrarojos destinados al estudio de lentes gravitacionales débiles y materia oscura en proyectos como WFIRST y Euclid.  

Con el programa de maestría pretendo continuar mi desarrollo como científico y concluir los programas de investigación que he comenzado. Además, mediante una asistencia graduada espero seguir adquiriendo experiencia como educador. De esta manera, me prepararé para continuar con estudios de doctorado fuera del país. Todo esto será con el fín de continuar una vida académica como profesor e investigador en Colombia. De esta manera podré apoyar el desarrollor científico de mi país y dar de mi parte para que logremos llegar a un mejor futuro.

\closing{Agradeciéndoles su atención,}

\end{letter}

\end{document}